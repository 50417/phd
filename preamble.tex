%
% LaTeX template for prepartion of submissions to PLDI'16
%
% Requires temporary version of sigplanconf style file provided on
% PLDI'16 web site.
% 
\documentclass[pageno,nonatbib]{jpaper}
% \documentclass[pldi-cameraready]{sigplanconf-pldi16}

\newcommand{\asplossubmissionnumber}{XXX}

\usepackage[utf8]{inputenc}

\usepackage[normalem]{ulem}

%
% the following standard packages may be helpful, but are not required
%
% \usepackage{siunitx}            % typset units correctly
% \usepackage{courier}            % standard fixed width font
% \usepackage[scaled]{helvet} % see www.ctan.org/get/macros/latex/required/psnfss/psnfss2e.pdf
% \usepackage{url}                  % format URLs
% \usepackage{listings}          % format code
% \usepackage{enumitem}      % adjust spacing in enums
% \usepackage[colorlinks=true,allcolors=blue,breaklinks,draft=false]{hyperref}   % hyperlinks, including DOIs and URLs in bibliography
% known bug: http://tex.stackexchange.com/questions/1522/pdfendlink-ended-up-in-different-nesting-level-than-pdfstartlink
% \newcommand{\doi}[1]{doi:~\href{http://dx.doi.org/#1}{\Hurl{#1}}}   % print a hyperlinked DOI

% Start of 'ignore natbib' hack
\let\bibhang\relax
\let\citename\relax
\let\bibfont\relax
\let\Citeauthor\relax
\expandafter\let\csname ver@natbib.sty\endcsname\relax
% End of 'ignore natbib' hack

% sort and compress citations
\usepackage[%
backend=biber,
style=numeric-comp,
sorting=none,        % nty,nyt,nyvt,anyt,anyvt,ynt,ydnt,none
sortcites=true,      % sort \cite{b a d c}: true,false
block=none,          % space between blocks: none,space,par,nbpar,ragged
indexing=false,      % indexing options: true,false,cite,bib
citereset=none,      % don't reset cites
isbn=false,          % print ISBN?
url=true,            % print URL?
doi=false,           % print DOI?
natbib=true,         % natbib compatability
maxbibnames=99       % no 'et-al' condensing author names in biliography
]{biblatex}

\addbibresource{./refs.bib}

\usepackage{graphicx}

% for subfloat:
\usepackage{subfig}

%\usepackage{csquotes}
%
%% no border around links
%\usepackage{xcolor}
%\hypersetup{
%    colorlinks,
%    linkcolor={black},
%    citecolor={black},
%    urlcolor={black}
%}
%
\usepackage{color}
\newcommand{\cec}[1]{{\color{red}{\emph{(Chris)} #1}}}
\newcommand\review[1]{\textcolor{blue}{\emph{(Review)} #1}}
%
%\usepackage[normalem]{ulem}
%


% Tables.
\usepackage{booktabs}
\usepackage{tabularx}
\usepackage{hhline}
\usepackage{xspace}
\usepackage[table]{xcolor}

% Define column types L, C, R with known text justification and fixed widths:
\usepackage{array}
\newcolumntype{L}[1]{>{\raggedright\let\newline\\\arraybackslash\hspace{0pt}}m{#1}}
\newcolumntype{C}[1]{>{\centering\let\newline\\\arraybackslash\hspace{0pt}}m{#1}}
\newcolumntype{R}[1]{>{\raggedleft\let\newline\\\arraybackslash\hspace{0pt}}m{#1}}

%% needed for \upgammar
%\usepackage{upgreek}
%
% Source code listings.
\usepackage{listings}
\lstset{%
  basicstyle=\scriptsize,%
  numbers=left,%
  xleftmargin=2em,
  framexleftmargin=2em,
  escapeinside={@|}{|@},
  frame=b,%
  breaklines=true,%
  postbreak=\raisebox{0ex}[0ex][0ex]{\ensuremath{\color{red}\hookrightarrow\space}},%
  % red arrow at line breaks
  captionpos=b%
}

% OpenCL listings
%
% From:
% http://gpumodeling.blogspot.com/2011/06/opencl-programs-in-latex-listings.html
\lstdefinelanguage[OpenCL]{C}[ANSI]{C}
{morekeywords={__kernel,kernel,__local,local,__global,global,%
    __constant,constant,__private,private,%
    char2,char3,char4,char8,char16,%
    uchar2,uchar3,uchar4,uchar8,uchar16,%
    short2,short3,short4,short8,short16,%
    ushort2,ushort3,ushort4,ushort8,ushort16,%
    int2,int3,int4,int8,int16,%
    uint2,uint3,uint4,uint8,uint16,%
    long2,long3,long4,long8,long16,%
    ulong2,ulong3,ulong4,ulong8,ulong16,%
    float2,float3,float4,float8,float16,%
    image2d_t,image3d_t,sampler_t,event_t,size_t,%
    bool2,bool3,bool4,bool8,bool16,%
    half2,half3,half4,half8,half16,%
    quad,quad2,quad3,quad4,quad8,quad16,%
    complex,imaginary},%
}%
%
%% Pseudo-code listings.
%\usepackage{algorithm}
%\usepackage{algorithmicx}
%\usepackage{algpseudocode}
%\usepackage[american]{babel}
%
%\newcommand\DeepTune{DeepTune\xspace}
%
%% math
% \usepackage{amssymb}
%\usepackage{bm}
%\usepackage{amsmath}
%\DeclareMathOperator*{\argmin}{arg\,min}
%\DeclareMathOperator*{\argmax}{arg\,max} 

\usepackage{tikz}
\def\checkmark{\tikz\fill[scale=0.2](0,.35) -- (.25,0) -- (1,.7) -- (.25,.15) -- cycle;}

\usepackage{amssymb}
\usepackage{pifont}
\usepackage{multirow}
\newcommand{\cmark}{\ding{51}}
\newcommand{\xmark}{\ding{55}}