\section{Bugs Discovered in OpenCL Compilers}

This section reports on the results of extensive testing of OpenCL compilers using CLgen. We found bugs in all the compilers we tested --- every compiler crashed, and every compiler silently generated wrong code. To date, we have submitted XX bug reports to compiler vendors. We divide our discussion by class of bug: first compiler and runtime crashes, then wrong-code bugs.

\subsection{Compiler Crashes}

We found numerous cases where a kernel triggers a crash in the compiler (and as a result, the host process). Figure~\ref{lst:intel-llvm-assertion} fails an assertion from the LLVM code generator backend. Segmentation faults during compilation were a common source of crash. Figure~\ref{lst:intel-vectorizer-segfault} segfaults during the vectorizer pass. Occasionally, the cause of the crash is unclear, such as in Figure~\ref{lst:nvidia-compile-segfault} which segfaults the compiler without warning.

In the majority of cases, we found compiler crashes to be insensitive to OpenCL optimization setting. Figure~\ref{lst:nvidia-compile-segfault} shows a rare case where a crash occurs only if optimizations are enabled.

\lstset{language=[OpenCL]C}
\begin{figure}
	\centering %
	\subfloat[Configs.\ $5\pm$ segementation fault in vectorizer pass.]{%
		\noindent\mbox{\parbox{\columnwidth}{%
				% bug-reports/coCLgenResults/segfaults/bug-report-intel-43729.c
				\lstinputlisting{lst/bug5}%
			}%
		}%
		\label{lst:intel-vectorizer-segfault}
	}\\%
	\subfloat[Configs.\ $10\pm$ compiler assertion during semantic analysis.]{%
		\noindent\mbox{\parbox{\columnwidth}{%
				% bug-reports/CLgenResults/bc/oclgrind-typos/bug-report-oclgrind-165430.c
				\lstinputlisting{lst/bug7}%
			}%
		}%
		\label{lst:oclgrind-sema-assertion}
	}\\%
	\subfloat[Configs.\ $10\pm$ compiler assertion when global size > 1.]{%
		\noindent\mbox{\parbox{\columnwidth}{%
				% bug-reports/CLgenResults/bc/oclgrind-densemap/bug-report-oclgrind-434050.c
				\lstinputlisting{lst/bug8}%
			}%
		}%
		\label{lst:oclgrind-llvm-densemap-assertion}
	}\\%
	\subfloat[Configs.\ $1\pm$ segementation fault.]{%
		\noindent\mbox{\parbox{\columnwidth}{%
				% bug-reports/coCLgenResults/segfaults/bug-report-nvidia-21742.c
				\lstinputlisting{lst/bug6}%
			}%
		}%
		\label{lst:nvidia-compile-segfault}
	}\\%
	\subfloat[Configs.\ $1\pm$, $2\pm$ segementation fault.]{%
		\noindent\mbox{\parbox{\columnwidth}{%
				% bug-reports/CLgenResults/bc/segfaults/bug-report-nvidia-561489.c
				\lstinputlisting{lst/bug11}%
			}%
		}%
		\label{lst:nvidia-recursion-segfault}
	}\\%
	\subfloat[Configs.\ $7\pm$ segementation fault.]{%
		\noindent\mbox{\parbox{\columnwidth}{%
				% bug-reports/coCLgenResults/segfaults/bug-report-intel-69539.c
				\lstinputlisting{lst/xeon-phi-segfault}%
			}%
		}%
		\label{lst:nvidia-compile-segfault}
	}%
	\caption{Example kernels which expose compiler crash bugs.}%
	\label{lst:compiler-crash-bugs}%
\end{figure}

\lstset{language=[OpenCL]C}
\begin{figure}
	\centering %
	\subfloat[Configs.\ $3\pm$ assertion during code generation for storing pointer values.]{%
		\noindent\mbox{\parbox{\columnwidth}{%
				% cl_launcherCLgenResults/runtime_crash/bug-report-intel-30870.sh
				\lstinputlisting{lst/bug1}%
			}%
		}%
		\label{lst:intel-llvm-assertion}
	}\\%
	\subfloat[Configs.\ $3\pm$ assertion during code generation for scalar types.]{%
		\noindent\mbox{\parbox{\columnwidth}{%
				% bug-reports/CLgenResults/bc/scalar-type/bug-report-intel-534566.c
				\lstinputlisting{lst/bug10}%
			}%
		}%
		\label{lst:intel-scalartype-assertion}
	}%
	\caption{Kernels which expose compiler back-end bugs.}%
	\label{lst:compiler-crash-bugs}%
\end{figure}

\subsection{Runtime Crashes}

Figure~\ref{lst:pocl-undefined-symbols} exposes a bug in in which a kernel containing an undefined symbol will successfully compile without warning, then crash the program when attempting to run the kernel. Such a test case is trivial to generate, yet could not be synthesized by CSmith.

\lstset{language=[OpenCL]C}
\begin{figure}
	\centering %
	\subfloat[Compilation should fail; config $6\pm$ succeeds then crashes program.]{%
		\noindent\mbox{\parbox{\columnwidth}{%
				\lstinputlisting{lst/bug2}%
			}%
		}%
		\label{lst:pocl-undefined-symbols}
	}\\%
	\caption{Example kernels which expose runtime crash bugs.}%
\label{lst:crash-bugs}%
\end{figure}


\subsection{Wrong-code Bugs}

\lstset{language=[OpenCL]C}
\begin{figure}
	\centering %
	\subfloat[Configs.\ $1+$, $2+$, $3-$, $4+$, $5-$, $6-$, $7+$, $8+$, $9-$, $10+$ incorrect vector width. \cc{TODO: Investigate testcase 32786}]{%
		\noindent\mbox{\parbox{\columnwidth}{%
				\lstinputlisting{lst/bug13}%
			}%
		}%
	}\\%
%SELECT 	results.id as 'results_id',
%device,
%num,
%testcase_id,
%params_id,
%gsize_x,
%optimizations,
%outcome,
%stdout_id
%FROM CLgenResults results
%INNER JOIN Testbeds ON results.testbed_id = Testbeds.id
%INNER JOIN Configurations ON results.testbed_id = Configurations.id
%INNER JOIN CLgenTestCases testcases ON results.testcase_id = testcases.id
%INNER JOIN cldriveParams params ON testcases.params_id = params.id
%WHERE program_id = 67869
%ORDER BY testcase_id, stdout_id, num;
	\subfloat[Configs.\ $4+$, $6+$ incorrectly execute \texttt{if} branch.]{%
	\noindent\mbox{\parbox{\columnwidth}{%
			\lstinputlisting{lst/CLgenProgram-67869}%
		}%
	}%
	\label{lst:intel-size_t-int}
}\\%
	\caption{Kernels for which compilers silently emit wrong-code.}%
\label{lst:compiler-crashes}%
\end{figure}
