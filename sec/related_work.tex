\section{Related Work} \label{sec:rw}

Generating OpenCL programs using syntactic level model~\cite{Cummins2017a}.

Random program generation is an effective method for software testing. Grammar-based \emph{fuzz testers} have been developed for C~\cite{Yang2012} and OpenCL~\cite{Lidbury2015a}. A mutation-based approach for the Java Virtual Machine is demonstrated in~\cite{Chena}. Goal-directed program generators have been used for a variety of domains, including generating linear transforms~\cite{Voronenko2009}, MapReduce programs~\cite{Smith}, and data structure implementations~\cite{Loncaric2016}. Program synthesis from input/output examples is used for simple algorithms in~\cite{Zaremba2015a}, string manipulation in~\cite{Gulwani2011}, and geometry constructions in~\cite{Gulwani2012}.

Machine learning has been applied to source code to aid software engineering. Naturalize employs techniques developed in the natural language processing domain to model coding conventions~\cite{Allamanis2014a}. JSNice leverages probabilistic graphical models to predict program properties such as identifier names for Javascript~\cite{Raychev}.

There is an increasing interest in mining source code repositories at large scale~\cite{Allamanis2013a,White2015a,Bird2009}. Previous studies have involved data mining of GitHub to analyze software engineering practices~\cite{Wu2014,Guzman2014,Baishakhi2014a,Vasilescu2015}, for example code generation~\cite{Zhang2015a}, code summarization~\cite{Allamanis2016}, comment generation~\cite{Wong2013}, and code completion~\cite{Raychev2014}. However, no work so far has exploited mined source code for benchmark generation. This work is the first to do so.