\section{Experimental Setup}

In this section we describe the particular experimental parameters used.

\subsection{Platforms}

\begin{table*}[t!]
  \scriptsize %
  \centering %
  % \rowcolors{2}{white}{gray!25}
  \begin{tabular}{rlllllllrr}
\toprule
 \#. &                                             Device &              Platform &    Driver & OpenCL &      Operating system &  Device type & Testing time &  B.R. Generated &  B.R. Submitted \\
\midrule
  1 &                                   GeForce GTX 1080 &           NVIDIA CUDA &    375.39 &    1.2 &    Ubuntu 16.04 64bit &          GPU &          30h &              13 &               7 \\
  2 &                                    GeForce GTX 780 &           NVIDIA CUDA &    361.42 &    1.2 &  openSUSE  13.1 64bit &          GPU &           0h &               0 &               0 \\
  3 &           Intel(R) HD Graphics Haswell GT2 Desktop &  Intel Gen OCL Driver &       1.3 &    1.2 &    Ubuntu 16.04 64bit &          GPU &           2h &              35 &              11 \\
  4 &          Intel(R) Xeon(R) CPU E5-2620 v4 @ 2.10GHz &          Intel OpenCL &  1.2.0.25 &    2.0 &    Ubuntu 16.04 64bit &          CPU &          10h &              10 &               5 \\
  5 &          Intel(R) Xeon(R) CPU E5-2650 v2 @ 2.60GHz &          Intel OpenCL &  1.2.0.44 &    1.2 &      CentOS 7.1 64bit &          CPU &           7h &               2 &               1 \\
  6 &            Intel(R) Core(TM) i5-4570 CPU @ 3.20GHz &          Intel OpenCL &  1.2.0.25 &    1.2 &    Ubuntu 16.04 64bit &          CPU &           1h &               4 &               4 \\
  7 &    Intel(R) Many Integrated Core Acceleration Card &          Intel OpenCL &       1.2 &    1.2 &      CentOS 7.1 64bit &  Accelerator &          11h &               0 &               0 \\
  8 &  pthread-Intel(R) Xeon(R) CPU E5-2620 v4 @ 2.10GHz &                  POCL &      0.14 &    2.0 &    Ubuntu 16.04 64bit &          CPU &          15h &             170 &              52 \\
  9 &                                 Oclgrind Simulator &              Oclgrind &     16.10 &    1.2 &    Ubuntu 16.04 64bit &     Emulator &           9h &               0 &               0 \\
\bottomrule
\end{tabular}

  \caption{%
    OpenCL platforms, the time spent in automated testing, and the number of bug reports submitted to date. \cc{refresh bugs submitted}%
  }
  \label{tab:platforms}
\end{table*}

We conducted testing of 10 OpenCL platforms, summarized in Table~\ref{tab:platforms}. Each testbed consists of a \emph{<device, driver>} pair. We covered a broad range of hardware: 3 GPUs, 4 CPUs, a co-processor, and an emulator. 7 of the compilers tested are commercial products, 3 of them are open source. Our suite of platforms includes both combinations of different drivers for the same device, and different devices using the same driver.

\subsection{Testbeds}
%HJL - Not sure if this is correct - if it is, ain't in the tables. So probably not.}
\hl{For each platform, we create multiple testbeds by various combinations of the following settings: running with or without compiler optimizations; running with a single thread or running with 2048 threads.}
%HJL alt- For each platform, we create two testbeds by either running with or without compiler optimizations.
