\section{Overview of Our Approach and Results}\label{sec:overview}

\paragraph{Test case generation} We mine GitHub for OpenCL fragments. We construct a corpus of example programs. We learn a generative model over these programs, and use a discriminative model to tune via feedback cycle. CLgen, our deep learning program synthesis tool, automatically generates an unbounded number of OpenCL samples for testing compilers.

\paragraph{OpenCL configurations tested} We conducted testing of 9 different OpenCL configurations, summarized in Table~\ref{tab:platforms}. A \emph{configuration} refers to an OpenCL \emph{<device, driver>} pair. We covered the broadest range of hardware available to us: 3 GPUs, 4 CPUs, an Accelerator, and an Emulator. 7 of the OpenCL implementations are commercial products, 2 of them are open source. We tested both different drivers for the same device, and different devices using the same driver.


\begin{table*}[t!]
	\scriptsize %
	\centering %
	% \rowcolors{2}{white}{gray!25}
	\begin{tabular}{rlllllllrr}
\toprule
 \#. &                                             Device &              Platform &    Driver & OpenCL &      Operating system &  Device type & Testing time &  B.R. Generated &  B.R. Submitted \\
\midrule
  1 &                                   GeForce GTX 1080 &           NVIDIA CUDA &    375.39 &    1.2 &    Ubuntu 16.04 64bit &          GPU &          30h &              13 &               7 \\
  2 &                                    GeForce GTX 780 &           NVIDIA CUDA &    361.42 &    1.2 &  openSUSE  13.1 64bit &          GPU &           0h &               0 &               0 \\
  3 &           Intel(R) HD Graphics Haswell GT2 Desktop &  Intel Gen OCL Driver &       1.3 &    1.2 &    Ubuntu 16.04 64bit &          GPU &           2h &              35 &              11 \\
  4 &          Intel(R) Xeon(R) CPU E5-2620 v4 @ 2.10GHz &          Intel OpenCL &  1.2.0.25 &    2.0 &    Ubuntu 16.04 64bit &          CPU &          10h &              10 &               5 \\
  5 &          Intel(R) Xeon(R) CPU E5-2650 v2 @ 2.60GHz &          Intel OpenCL &  1.2.0.44 &    1.2 &      CentOS 7.1 64bit &          CPU &           7h &               2 &               1 \\
  6 &            Intel(R) Core(TM) i5-4570 CPU @ 3.20GHz &          Intel OpenCL &  1.2.0.25 &    1.2 &    Ubuntu 16.04 64bit &          CPU &           1h &               4 &               4 \\
  7 &    Intel(R) Many Integrated Core Acceleration Card &          Intel OpenCL &       1.2 &    1.2 &      CentOS 7.1 64bit &  Accelerator &          11h &               0 &               0 \\
  8 &  pthread-Intel(R) Xeon(R) CPU E5-2620 v4 @ 2.10GHz &                  POCL &      0.14 &    2.0 &    Ubuntu 16.04 64bit &          CPU &          15h &             170 &              52 \\
  9 &                                 Oclgrind Simulator &              Oclgrind &     16.10 &    1.2 &    Ubuntu 16.04 64bit &     Emulator &           9h &               0 &               0 \\
\bottomrule
\end{tabular}

	\caption{OpenCL configurations we tested, the total time spent running tests, and the number of bug reports submitted to date.}
	\label{tab:platforms}
\end{table*}


\paragraph{Bugs found} All configurations yielded bugs. Every compiler crashed, and every compiler silently generated wrong code. To date, we have submitted XX bug reports to compiler vendors. In comparing our approach against the state-of-the-art in OpenCL compiler test case generation, we found our approach was able to \cc{\ldots}