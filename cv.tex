% Chris Cummins Curriculum Vitae
%
% Based on "Classicthesis-Styled CV" by Alessandro Plasmati,
% downloaded from: http://www.LaTeXTemplates.com
%
% License: CC BY-NC-SA 3.0 (http://creativecommons.org/licenses/by-nc-sa/3.0/)
%
\documentclass[a4paper,11pt,hidelinks]{scrartcl}

% Hide the date:
\date{}

% Move the margin to the left of the page:
\reversemarginpar

% New command defining the margin text style:
\newcommand{\MarginText}[1]{\marginpar{\raggedleft\itshape\normalsize#1}}

% Use the classicthesis style for the style of the document:
\usepackage[nochapters]{classicthesis}
% Use the currvita style for the layout of the document:
\usepackage[LabelsAligned]{currvita}

% Font color of your name at the top:
\definecolor{namecolour}{rgb}{.85,0,0}
\definecolor{titlecolour}{rgb}{0,0,0}

\renewcommand{\cvheadingfont}{\LARGE\color{namecolour}}

\usepackage{hyperref}
% Set link colors:
% \hypersetup{colorlinks, breaklinks, urlcolor=Red, linkcolor=Red}

% Set the width of the date box in each block:
\newlength{\datebox}\settowidth{\datebox}{Spring 2011}

% Set page margins:
\usepackage[top=1.4cm, bottom=1cm, left=4.3cm, right=2cm]{geometry}

% New section title:
\newcommand{\Title}[1]{\noindent\rule{15cm}{0.4pt}\vspace{-.45em}\\
\noindent\spacedlowsmallcaps{\color{titlecolour}{#1}}

\vspace{-.85em}\noindent\rule{15cm}{0.4pt}\vspace{.2em}}

% Description blocks:
\newcommand{\Description}[1]{\hangindent=0em\hangafter=0%
\noindent\normalsize{#1}\vspace{1em}}

\newcommand{\ShortJob}[3]{\Description{\MarginText{#1}\textbf{#2},
  \textit{#3}}}

\newcommand{\Job}[4]{\Description{\MarginText{#1}\textbf{#2},
  \textit{#3}\newline\vspace{-.8em}

  \noindent
  #4}}


\newcommand{\Publication}[5]{\Description{\MarginText{#1}#2. \textbf{\href{#5}{#3}}. #4.}}

\newcommand{\Award}[2]{\Description{\MarginText{#1}#2}
\vspace{-.9em}}

\begin{document}

% Stop the page count at the bottom of the first page:
\thispagestyle{empty}


%%%%%%%%%%
% HEADER %
%%%%%%%%%%
\begin{cv}%
%
{\spacedallcaps{Chris Cummins}} %

\begin{flushright}
\vspace{-3.1em}
\hfill \noindent \href{mailto:chrisc.101@gmail.com}{\textsc{chrisc.101@gmail.com}}\\%
\vspace{-.35em}
\hfill \noindent \href{http://chriscummins.cc}{\textsc{http://chriscummins.cc}}\\%
\end{flushright}
\vspace{-1em}

%%%%%%%%%%%%%
% EDUCATION %
%%%%%%%%%%%%%
\Title{Education}

\Description{\MarginText{2019\\(expected)}\textbf{Ph.D, Informatics}
  \newline University of Edinburgh, School of Informatics
  \newline\vspace{-.8em}

  \noindent %
  Deep learning over programs. Developing machine learning methods for random program generation, compiler optimisations, and representative benchmarking. Applications for heterogeneous parallelism, testing, and adaptive performance tuning. To date: 3 publications, 5 posters, 4 conferences, 3 invited talks.%
}

\Description{\MarginText{2015}\textbf{MSc by Research, Pervasive Parallelism}
  \textit{(Distinction)}
  \newline University of Edinburgh, School of Informatics
  \newline Thesis: \textit{Autotuning Stencil Codes with Algorithmic
    Skeletons} (grade: 85\%)
  \newline\vspace{-.8em}

  \noindent %
  Runtime adaptive tuning for heterogeneous parallelism, achieving $3.79\times$ speedup of multi-GPU stencil programs. Machine learning over distributed training sets. High-level GPGPU programming. Published in \textit{HLPGPU '16} and \textit{ADAPT '16}.%
}

\Description{\MarginText{2014}\textbf{MEng Electronic Engineering \& Computer Science}
  \textit{(First Class Honours)}
  \newline Aston University, School of Engineering \& Applied Science
  \newline Thesis: \textit{Protein Isoelectric Point Database} (grade: 90\%)
  \newline\vspace{-.8em}

  \noindent %
  Created a novel search engine and research tool for molecular biochemistry. Developed full integration of BLAST search tools, a publicly accessible API, and tooling to generate synthetic payloads from confidential datasets for whitebox systems testing. Released open source. Published in \textit{Bioinformatics}.%
}

%%%%%%%%%%%%%%%%
% PUBLICATIONS %
%%%%%%%%%%%%%%%%
\Title{Publications}
% Format: \Publication{year}{authors}{title}{location}{url}

\Description{\MarginText{\textit{(in preparation)}}C.\ \textsc{Cummins}, P.\
  \textsc{Petoumenos}, Z.\ \textsc{Wang}, H.\
  \textsc{Leather}. \textbf{End-to-end Deep Learning of Compiler Heuristics}. To be submitted to PACT'17.\@
  \newline\vspace{-.8em}

  \noindent %
  Exceeds performance of state-of-the art predictive models using hand crafted features.%
}

\Description{\MarginText{2017}C.\ \textsc{Cummins}, P.\
  \textsc{Petoumenos}, Z.\ \textsc{Wang}, H.\
  \textsc{Leather}. \textbf{Synthesizing Benchmarks for Predictive Modeling}. Best Paper CGO'17 (22\% acceptance rate), Austin, Texas.\@
  \newline\vspace{-.8em}

  \noindent %
  Deep learning over massive codebases from GitHub to generate benchmark programs. Automatically synthesizes OpenCL kernels which are indistinguishable from hand-written code, and improves state-of-the-art predictive model performance by $4.30\times$.%
}

\Description{\MarginText{2016}C.\ \textsc{Cummins}, P.\
  \textsc{Petoumenos}, M.\ \textsc{Steuwer}, H.\
  \textsc{Leather}. \textbf{Towards Collaborative Performance Tuning
    of Algorithmic Skeletons}. HLPGPU'16, HiPEAC, Prague.\@
  \newline\vspace{-.8em}

  \noindent %
  An extensible and distributed framework for dynamic prediction of optimisation parameters at runtime. \emph{OmniTune} provides a flexible API to enable predictive autotuning with machine learning, automatically exceeding human experts by $1.22\times$.%
}

\Description{\MarginText{2016}C.\ \textsc{Cummins}, P.\
  \textsc{Petoumenos}, M.\ \textsc{Steuwer}, H.\
  \textsc{Leather}. \textbf{Autotuning OpenCL Workgroup Size for
    Stencil Patterns}. ADAPT'16, HiPEAC, Prague.\@ \newline\vspace{-.8em}

  \noindent %
  Three methodologies to autotune stencil patterns using machine learning classification and regression. We demonstrate a median $3.79\times$ speedup over the best possible fixed workgroup size, achieving 94\% of the maximum performance.%
}

\Description{%
  % Year:
  \MarginText{2015}%
  % Authors:
  E.\ \textsc{Bunkute}, %
  C.\ \textsc{Cummins}, %
  F.\ \textsc{Crofts}, %
  G.\ \textsc{Bunce}, %
  I.\ T.\ \textsc{Nabney}, %
  D.\ R.\ \textsc{Flower}.
  % Title:
  \textbf{\href{http://bioinformatics.oxfordjournals.org/content/31/2/295.full?etoc}{PIP-DB:
      The Protein Isoelectric Point Database}}.
  % Appears in:
  Bioinformatics, 31(2), 295-296. Chicago. %
  \newline\vspace{-.8em}

  % Blurb:
  \noindent %
  An open source search engine of protein isoelectric points. Provides public access to bioinformatics data from the literature for comparison and benchmarking purposes.%
}


\newpage

%%%%%%%%%%%%%%%%%%%%%%%%%%%
% PROFESSIONAL EXPERIENCE %
%%%%%%%%%%%%%%%%%%%%%%%%%%%
\Title{Professional Experience}
% Format: \Job{dates}{employer}{title}{description}
% Format: \ShortJob{dates}{employer}{title}

\Description{\MarginText{2016}\textbf{Codeplay Software}
  \newline Software Engineer Intern, Eigen SYCL Interface
  \newline\vspace{-.8em}

  \noindent %
  Developing OpenCL port of Tensorflow. Implemented GPU memory management for Eigen. Compile time scheduling and kernel fusion for expression trees on GPUs. Proposed and implemented a Python interface for VisionCpp. Extensive C++ meta-programming.%
}

\Description{\MarginText{2012--2013}\textbf{Intel Corporation}
  \newline Open Source Developer Intern
  \newline\vspace{-.8em}

  \noindent %
  Patched \texttt{ioctl} subsystem in Linux kernel. Developed a novel SIMD register visualisation tool for Intel GPU assembly programming. Implemented GTK+ support for Wayland display server. Fixed memory and usability bugs in GNOME desktop applications. Developed particle effects engine for a 3D rendering program. Rapid prototyping of Android applications. Numerous contributions to open source projects.%
}

\Description{\MarginText{2010--2014}\textbf{Freelance}
  \newline Web Developer
  \newline\vspace{-.8em}

  \noindent %
  Full-stack development for small businesses, including graphic design and branding. Front-end experience with JavaScript; back-end development using Clojure, Node.js, PHP, MySQL, PostgreSQL, and Jekyll. Clients have included publishing companies, musicians, and a beauty parlour.%

  % Clients:
  %
  % The Beauty Rooms
  % Barthelemy Jusselme
  % Myrmidon Books
  % Border Scripts
  % Greystones Press
}

\Description{\MarginText{2008}\textbf{Rolls Royce Holdings plc}
  \newline Work placement in the Design Methods \& Improvements team.%
}


%%%%%%%%%%
% AWARDS %
%%%%%%%%%%
\Title{Awards}
% Format: \Award{year}{title}

\Description{\MarginText{2017}\textbf{Best Paper Winner, CGO} \newline\vspace{-1.8em}}

\Description{\MarginText{2015}\textbf{PhD studentship, EPSRC grant
    EP/L01503X/1} \newline\vspace{-1.8em}}

\Description{\MarginText{2014}\textbf{Institute of Engineering \&
    Technology Prize}
  \newline\vspace{-.8em}

  \noindent Annual prize for top engineering student at Aston University.%
}

\Description{\MarginText{2009}\textbf{Arkwright Scholarship, Rolls
    Royce Holdings plc}
  \newline\vspace{-.8em}

  \noindent Funded scholarship awarded to less than 250 students nationwide.%
}

\Description{\MarginText{2009}\textbf{Engineering Education Scheme of
    England}
  \newline\vspace{-.8em}

  \noindent R\&D for a (now patented) supermarket trolley mounted shopping aid.%
}

\Description{\MarginText{2008}\textbf{AESSEAL Design Innovation Award}
  \newline\vspace{-.8em}

  \noindent Cash prize for first place in an industrial 3D CAD
  competition.%
}

%%%%%%%%%%%%%%%%%%%%%%%
% ACADEMIC ACTIVITIES %
%%%%%%%%%%%%%%%%%%%%%%%
\Title{Academic Activities}

\Description{%
  \MarginText{\textbf{Invited Talks}}%
  Codeplay Software 2016, %
  Ocado Technology 2016, %
  Amazon Development Center 2016.%
  \vspace{-.8em}}

\Description{%
  \MarginText{\textbf{Posters}}%
  Google 2016, %
  ACACES 2016, %
  PLDI 2016, %
  HiPEAC 2016, %
  Google 2015, %
  PPar 2015.%
  \vspace{-.8em}}

\Description{%
  \MarginText{\textbf{Peer reviews}}%
  ACM TACO 2016, %
  LCTES 2016, %
  CGO 2016.%
  \vspace{-.8em}}

% \Description{\MarginText{\textbf{Volunteering}}%
%   ParCo 2015.%
%   \vspace{-.8em}}

\vspace{.8em}


%%%%%%%%%%%%%%%%%%%%
% Technical skills %
%%%%%%%%%%%%%%%%%%%%
\Title{Technical Skills}

\Description{\MarginText{\textbf{Expert}}%
  C++, Python, bash, git, GNU/Linux.%
  \vspace{-.8em}%
}

\Description{\MarginText{\textbf{Advanced}}%
  C, JavaScript, OpenCL, SYCL, SQL, \LaTeX, GNU autotools, gdb.%
  \vspace{-.8em}%
}

% \Description{\MarginText{\textbf{Competent}}%
%   Java, Clojure LISP, Lua, MATLAB, CMake, PHP, VHDL, bazel, CMake, x86 assembly.%
%   \vspace{-.8em}%
% }

\end{cv}

\end{document}
