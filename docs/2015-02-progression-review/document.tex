% A two page document describing lessons learned from the project, the
% revised PhD vision and insights, and a plan of activities for the
% next three months
%
\documentclass[11pt]{article}

\author{Chris Cummins}
\date{September 2015}
\title{PhD Progression Review: Autotuning\\and Skeleton-aware Compilation}

%%%%%%%%%%%%%%%%%%%%%%%%%
%% Document and Layout %%
%%%%%%%%%%%%%%%%%%%%%%%%%

% Fix for multiple "No room for a new \dimen" errors.
%
% See: http://tex.stackexchange.com/questions/38607/no-room-for-a-new-dimen
%
\usepackage{etex}

\usepackage[utf8]{inputenc}

% Fix for "'babel/polyglossia' detected but 'csquotes' missing"
% warning. NOTE: Include after inputenc.
%
\usepackage{csquotes}

% Make internal macro definitions accessible,
% e.g. \@title, \@date \@author.
\makeatletter

% Multi-column support.
\usepackage{multicol}

% A useful package which includes macros like \ifdef{}{}{}:
%
\usepackage{etoolbox}

% Uncomment the following line to remove column separation:
%
%\setlength{\columnsep}{5mm}

% Allow user-defined warning and error filters.
%
\usepackage{silence}


%%%%%%%%%%%%%%%%%%%%%
% Table of Contents %
%%%%%%%%%%%%%%%%%%%%%

% % Set chapter and section numbering depth:
% %
% \setcounter{secnumdepth}{2}


%%%%%%%%%%%%%%%%
% Bibliography %
%%%%%%%%%%%%%%%%
\usepackage[%
    backend=biber,
    style=numeric-comp,
    % style=numeric-comp,  % numerical-compressed
    sorting=none,        % nty,nyt,nyvt,anyt,anyvt,ynt,ydnt,none
    sortcites=true,      % sort \cite{b a d c}: true,false
    block=none,          % space between blocks: none,space,par,nbpar,ragged
    indexing=false,      % indexing options: true,false,cite,bib
    citereset=none,      % don't reset cites
    isbn=false,          % print ISBN?
    url=true,            % print URL?
    doi=false,           % print DOI?
    natbib=true,         % natbib compatability
  ]{biblatex}

% % Filter annoying and unavoidable biblatex warning:
\WarningFilter{biblatex}{Patching footnotes failed}

% Reduce the font size of the bibliography:
% \renewcommand{\bibfont}{\normalfont\scriptsize}

% Determine which BibTeX file to use:
%
% If available, use my Mendeley BibTex library, located in the home
% directory. Note that this is a relative path and will break if
% either this file or the BibTex library are moved. If the library is
% not present, use the local refs.bib file.
\newcommand{\BibResourceGlobal}{../../library.bib}
\newcommand{\BibResourceLocal}{refs.bib}

\IfFileExists{\BibResourceGlobal}
  {\newcommand{\BibResource}{\BibResourceGlobal}}
  {\newcommand{\BibResource}{\BibResourceLocal}}

\addbibresource{\BibResource}


%%%%%%%%%%%%%%
% Appendices %
%%%%%%%%%%%%%%

% Appendix package. Documentation:
%
%  http://mirror.ox.ac.uk/sites/ctan.org/macros/latex/contrib/appendix/appendix.pdf
%
% Package options:
%
% toc      - Put a header (e.g., `Appendices') into the Table of Contents
%            (the ToC) before listing the appendices. (This is done by
%            calling the \addappheadtotoc command.)
% page     - Puts a title (e.g., `Appendices') into the document at the
%            point where the appendices environment is begun. (This is
%            done by calling the \appendixpage command.)
% title    - Adds a name (e.g., `Appendix') before each appendix title in
%            the body of the document. The name is given by the value
%            of \appendixname. Note that this is the default behaviour
%            for classes that have chapters.
% titletoc - Adds a name (e.g., `Appendix') before each appendix listed
%            in the ToC. The name is given by the value
%            of \appendixname.
% header   - Adds a name (e.g., `Appendix') before each appendix in page
%            headers.  The name is given by the value
%            of \appendixname. Note that this is the default behaviour
%            for classes that have chapters.
\usepackage[title, titletoc]{appendix}


%%%%%%%%%%%%%%%%%%%%%%%%%%%%%%%%%%%%%
%% Figures, footnotes and listings %%
%%%%%%%%%%%%%%%%%%%%%%%%%%%%%%%%%%%%%

%\usepackage{float}
%\restylefloat{figure}

% Use bold ``(Figure|Table|Listing)'' caption text.
%\usepackage[margin=1cm]{caption}

% Set the font for captions.
% \renewcommand{\captionfont}{\small}
% Set the font for caption labels.
% \renewcommand{\captionlabelfont}{\footnotesize\bf}

% Use arabic numbers for footnote.
%\renewcommand{\thefootnote}{\arabic{footnote}}

% Ensure that footnotes always appear at the bottom of pages.
%\usepackage[bottom]{footmisc}

% Reset the footnote counter on every page.
%\usepackage{perpage}
%\MakePerPage{footnote}

% Pre-requisites for rendering upquotes in listings package.
\usepackage[T1]{fontenc}
\usepackage{lmodern}
\usepackage{textcomp}

% Pseudo-code listings.
\usepackage{algorithm}
\usepackage{algpseudocode}
\newcommand{\Break}{\State \textbf{break} }
\algblockdefx[Loop]{Loop}{EndLoop}[1][]{\textbf{Loop} #1}{\textbf{End
    Loop}}

\algrenewcommand\ALG@beginalgorithmic{\footnotesize}

% Code listings.
\usepackage{listings}

% Set \ttfamily to use courier fonts.
%
% See: http://tex.stackexchange.com/a/33686
%
\usepackage{courier}

\lstset{frame=bt,                    % Add top and bottom frame lines
        breaklines=true,             % Force line wrapping
        captionpos=b,                % Place caption below listing
        numbers=left,                % Add left-side line numbers
        basicstyle=\scriptsize\ttfamily, % Set font size and type
        showstringspaces=false,      % Don't show visible whitespace
        numberstyle=\tiny,
        upquote=true,                % Use upright quotes, not curly
        commentstyle=\bfseries}      % Embolden comments

% Use (*@ @*) to escape LaTeX commands within listings.
\lstset{escapeinside={(*@}{@*)}}

% Add 10pt space between chapters in TOC listings entries:
%\let\Chapter\chapter
%\def\chapter{\addtocontents{lol}{\protect\addvspace{10pt}}\Chapter}


%%%%%%%%%%%%%%%%%%%%%%%%
%% Graphics and maths %%
%%%%%%%%%%%%%%%%%%%%%%%%
\usepackage{amsmath}

% Vector notation, e.g. \vv{x}:
%
\usepackage{esvect}

% Additional amsmath symbols, see:
%
% http://texblog.org/2007/08/27/number-sets-prime-natural-integer-rational-real-and-complex-in-latex/
%
\usepackage{amsfonts}
\usepackage{amssymb}

\usepackage{graphicx}
\usepackage{mathtools}
\usepackage{tikz}
\usepackage{tikz-qtree}

% Provide bold font face in maths.
\usepackage{bm}

\usepackage{subcaption}
\expandafter\def\csname ver@subfig.sty\endcsname{}

% Define an 'myalignat' command which behave as 'alignat' without the
% vertical top and bottom padding. See:
%     http://www.latex-community.org/forum/viewtopic.php?f=5&t=1890
\newenvironment{myalignat}[1]{%
  \setlength{\abovedisplayskip}{-.7\baselineskip}%
  \setlength{\abovedisplayshortskip}{\abovedisplayskip}%
  \start@align\z@\st@rredtrue#1
}%
{\endalign}

% Define additional operators:
\DeclareMathOperator*{\argmin}{arg\,min}
\DeclareMathOperator*{\argmax}{arg\,max}

\DeclareMathOperator*{\gain}{Gain}

% Skeleton operators.
\DeclareMathOperator*{\map}{Map}
\DeclareMathOperator*{\reduce}{Reduce}
\DeclareMathOperator*{\scan}{Scan}
\DeclareMathOperator*{\stencil}{Stencil}
\DeclareMathOperator*{\zip}{Zip}
\DeclareMathOperator*{\allpairs}{All\,Pairs}

% Maths plots using pgfplots, see:
%
%     http://pgfplots.sourceforge.net/pgfplots.pdf
%
\usepackage{pgfplots}

% Disable compatability mode.
%
\pgfplotsset{compat=1.12}

% Gantt charts using pgfgantt, see:
%
%     http://www.ctan.org/pkg/pgfgantt
%
\usepackage{pgfgantt}

% Fix milestone aspect ratio by defining a custom element.
\newganttchartelement*{mymilestone}{
  mymilestone/.style={
    shape=diamond,
    inner sep=2pt,
    draw=black,
    top color=black,
    bottom color=black,
  }
}

% Tikz flowchart configuration.
\usetikzlibrary{shapes,arrows,shadows,fit,backgrounds}
\tikzstyle{decision} = [diamond,
                        draw,
                        text width=4.5em,
                        text badly centered,
                        node distance=3cm,
                        inner sep=0pt]
\tikzstyle{block}    = [rectangle,
                        draw,
                        text width=5em,
                        text centered,
                        node distance=3cm,
                        minimum height=4em,
                        inner sep=.2cm]
\tikzstyle{line}     = [draw, -latex']

% Add dirtree picture style, see:
%
%     http://tex.stackexchange.com/a/34268
%
\newcount\dirtree@lvl
\newcount\dirtree@plvl
\newcount\dirtree@clvl
\def\dirtree@growth{%
  \ifnum\tikznumberofcurrentchild=1\relax
    \global\advance\dirtree@plvl by 1
    \expandafter\xdef\csname dirtree@p@\the\dirtree@plvl\endcsname{\the\dirtree@lvl}
  \fi
  \global\advance\dirtree@lvl by 1\relax
  \dirtree@clvl=\dirtree@lvl
  \advance\dirtree@clvl by -\csname dirtree@p@\the\dirtree@plvl\endcsname
  \pgf@xa=0.33cm\relax
  \pgf@ya=-\baselineskip\relax
  \pgf@ya=\dirtree@clvl\pgf@ya
  \pgftransformshift{\pgfqpoint{\the\pgf@xa}{\the\pgf@ya}}%
  \ifnum\tikznumberofcurrentchild=\tikznumberofchildren
    \global\advance\dirtree@plvl by -1
  \fi
}
\tikzset{
  dirtree/.style={
    growth function=\dirtree@growth,
    every node/.style={anchor=north},
    every child node/.style={anchor=west},
    edge from parent path={(\tikzparentnode\tikzparentanchor) |- (\tikzchildnode\tikzchildanchor)}
  }
}

% UML sequence diagram macros, see:
%
%     https://code.google.com/p/pgf-umlsd/
%
% Options:
%
%     underline - Underline object names
%
\usepackage[underline=false]{pgf-umlsd}

% Support for SVG graphics.
%
% NOTE that you must pass the "--shell-escape" argument to pdflatex to
% compile. NOTE also that images *MUST* be placed within the graphics
% path.
\usepackage{svg}
\graphicspath{{img/}}

%%%%%%%%%%%%%%%%%%%%%%
%% Tables and lists %%
%%%%%%%%%%%%%%%%%%%%%%

% Required to use labm8 exported tables.
%
\usepackage{booktabs}

% Required for full page-width tables.
\usepackage{tabularx}

%\usepackage{enumitem}
%\setenumerate{itemsep=0pt}

% Use no left margin for lists:
%\setlist{leftmargin=*}

\usepackage{longtable}

% Define column types L, C, R with known text justification and fixed
% widths:
\usepackage{array}
\newcolumntype{L}[1]{>{\raggedright\let\newline\\\arraybackslash\hspace{0pt}}m{#1}}
\newcolumntype{C}[1]{>{\centering\let\newline\\\arraybackslash\hspace{0pt}}m{#1}}
\newcolumntype{R}[1]{>{\raggedleft\let\newline\\\arraybackslash\hspace{0pt}}m{#1}}


%%%%%%%%%%%%%%%%%%%%%%%%%%%%%
%% Typesetting and symbols %%
%%%%%%%%%%%%%%%%%%%%%%%%%%%%%

% Adjustable font sizes in \Verbatim{}
\usepackage{fancyvrb}

%\usepackage{titlesec}
% Set section and paragraph heading fonts:
%\titleformat*{\section}{\Large\bfseries}
%\titleformat*{\subsection}{\normalsize\bfseries}
%\titleformat*{\subsubsection}{\normalsize}
%\titleformat*{\paragraph}{\large\bfseries}
%\titleformat*{\subparagraph}{\large\bfseries}

% Set section heading margins. Usage:
% \titlespacing*{<command>}{<left>}{<before>}{<after>}
%\titlespacing*{\section}{0pt}{.6em}{.3em}
%\titlespacing*{\subsection}{0pt}{.6em}{.2em}

% Set paragraph indentation size. Default is 15pt.
%\setlength{\parindent}{10pt}

% The line spacing can be globally set using \linespread:
%
% \linespread{1.2}

% Add a command \hr{} which will draw a horizontal rule the width of
% the text.
%
\newcommand{\hr}{\noindent\makebox[\linewidth]{\rule{\textwidth}{0.2pt}}}

% Add a command \br{} which will create a horizontal space of exactly
% one line height.
%
\newcommand{\br}{\hspace{\baselineskip}}

% Define a command to allow word breaking.
\newcommand*\wrapletters[1]{\wr@pletters#1\@nil}
\def\wr@pletters#1#2\@nil{#1\allowbreak\if&#2&\else\wr@pletters#2\@nil\fi}

% Define a command to create centred page titles.
\newcommand{\centredtitle}[1]{
  \begin{center}
    \large
    \vspace{0.9cm}
    \textbf{#1}
  \end{center}}

% Support hyperlinks using the \hyperref, \url and \href
% macros. Usage:
%
%    \hyperref[label_name]{''link text''}
%
%    \url{<my_url>}
%
%    \href{<my_url>}{<description>}
%
\usepackage{hyperref}

% Disable colored borders of links, cross-references etc in PDF output
\hypersetup{pdfborder={0 0 0}}

% Provide generic commands \degree, \celsius, \perthousand, \micro
% and \ohm which work both in text and maths mode.
\usepackage{gensymb}

%%%%%%%%%%%%%%%%%%%%%%%%%%%%%%%%%
%% Placeholder text generation %%
%%%%%%%%%%%%%%%%%%%%%%%%%%%%%%%%%

% Use either \blindtext or \libpsum to generate placeholder text. Also
% note the macros \blinditemize, \blindenumerate, \blinddescription.
\usepackage[english]{babel}
\usepackage{blindtext}
\usepackage{lipsum}


\begin{document}

\begin{center}
\Large
\textbf{PhD Progression Review: Autotuning\\and Skeleton-aware Compilation}
\vspace{0.2cm}

\normalsize
Chris Cummins\\
September 2015
\vspace{0.2cm}
\end{center}
%\maketitle

\begin{abstract}
\noindent
I present a research proposal to develop an optimising compiler
specifically tailored for algorithmic skeletons. Such a compiler would
be capable of performing optimisations which are not possible by
library-level skeleton frameworks: optimisations across skeleton calls
(e.g.\ to re-order the execution of nested skeletons), and across
muscle functions (e.g.\ to perform load balancing of pipeline
stages). This document briefly outlines the proposed project, and
summarises the findings from the MSc phase of my CDT.
\end{abstract}

\section{Introduction}

In my initial CDT application, I postulated that in order to address
the under-utilisation of parallel resources that is a symptom of the
modern hardware/software ecosystem, the goal should be to move towards
\emph{automated} parallelism. My MSc thesis worked towards this goal
by developing a system for adaptive tuning of patterns (algorithmic
skeletons) for high level parallel programming. Skeletons provide
abstractions that free developers from the concerns of managing
parallel resources. The need for adaptive tuning of skeletons was
demonstrated by the significant performance improvements which were
achieved through autotuning of only a single two-dimensional parameter
space for stencil skeletons. I demonstrated a median geometric speedup
of $3.79\times$ across benchmarks from both real world and
synthetically generated sources. My intention over the coming months
is to broaden the scope of this work to consider the interaction
between algorithmic skeletons and optimising compilers.

% The first year of this CDT was spent investigating autotuning
% algorithmic skeletons for stencil codes on CPUs and GPUs. This was a
% challenging but rewarding project which providing me with a
% grounding in research approaches to high level parallel programming,
% GPU parallelism, autotuning, and machine learning. In this document
% I critically evaluate my work during this MSc phase, and propose a
% direction for future research which builds upon the knowledge and
% skills acquired in the first year, while addressing the challenges I
% faced.

% The broad scope of the proposal for my initial application was to
% address the parallel programmability challenge, by enabling a shift
% towards automatic parallelism. During my first three months the scope
% of this proposal was adjusted to focus on improving the high level
% parallel programming by adaptive tuning. To this end, my MSc thesis
% investigated adaptive tuning of a parameter space for skeletal
% programming. It is my intention to continue working towards this goal
% by focusing next on the interaction between compilers and algorithmic
% skeletons.

% I’ve shown that the optimisation space of a fixed implementation can
% be successfully tuned using machine learning. The next step is to
% combine these techniques with an understanding of optimising
% compilers in order to generate adaptive implementations.


\section{MSc Phase Review}

My MSc thesis introduced \textit{OmniTune}, a framework for runtime
autotuning of parameters. The effectiveness of this approach was
demonstrated using the workgroup size of stencil skeletons for CPUs
and GPUs. Machine learning models trained using data collected from
synthetically generated benchmarks were evaluated for prediction
quality in this two-dimensional parameter space.

The plan for the project consisted of three sequential phases:
identifying a profitable optimisation space, probing the space to
gather empirical performance data, and developing a system to exploit
this optimisation space. In reality, there was a great overlap between
the progress of each phase: identifying the optimisation space was
achieved using empirical performance data, and in turn gathering
performance data required the partial development of the autotuning
system. Broadly, the stencil optimisations space had been identified
by early March, the collection of empirical performance data began in
mid April, and the development of the final OmniTune autotuning system
began in mid May. For future work, a methodology based upon shorter,
iterative cycles of testing, implementing, and evaluating should be
used. This should also be coupled with regular write-ups of current
findings and results, in addition to the logbook I kept this
year. Another insight which can be transferred to future work is the
importance of clear, testable hypotheses, which should be identified
during early planning
stages. % All source code and notes were hosted publicly on GitHub.

% - short development cycles, with milestones and reviews at each stage
%
% - a more clearly testable hypothesis - i.e. a less vague project plan


\section{%Beginning the
PhD Phase}

While the tuning of parameter values demonstrably has a significant
impact on the performance of skeleton programs, there is great scope
for optimisation at the compilation stage. The complexity of parallel
programs restricts the number of optimisations which compilers may
apply, and furthermore there are higher-level restructurings and
implementation choices which could be made by compilers equipped with
an understanding of the semantics of skeleton operations.

\subsection{Objectives}

The objective of compiler research for algorithmic skeletons would be
to enable ``skeleton-aware compilation''. This can be considered both
as the enabling of existing compiler optimisations to algorithmic
skeletons which cannot currently be applied due to limitations in the
static analysis of parallel programs, and the development of novel
optimisations which are specific to algorithmic skeletons. In both
cases, skeleton-aware compilation will be able to provide measurable
performance improvements of programs compiled using the existing state
of the art in optimising compilers.


\subsection{Methodology}

This research project will begin with an exposition of the relevant
literature to help enumerate the range of possible optimisations which
could be applied, and to identify what has already been achieved.
When identifying a range of optimisations which could be developed,
low cost proof-of-concept tests can be performed for each by manually
performing an optimisation by hand for a specific benchmark and
gathering empirical performance data. If the optimisation leads to
measurable performance improvements, a transformation pass in a
relevant compiler (e.g.\ LLVM) can be implemented to automatically
apply this optimisation. Test programs may be selected from existing
benchmark suites (e.g. the Intel Thread Building Blocks port of
PARSEC), or by using synthetically generated programs, perhaps by
extending the benchmark generator developed for my MSc project. In
case of optimisations for which the potential benefit cannot
statically be determined (e.g.\ load balancing between stages of a
pipeline), the OmniTune framework may be extended to provide
autotuning capabilities for the space of possible optimisations.


\section{Summary}

In my first year I have demonstrated the importance of adaptive tuning
for maximising the performance of high level parallel patterns. I
intend to continue improving the performance of high level parallel
patterns, but taking an approach from the level of the compiler, with
the aim of developing skeleton-aware compilation. The skills I
acquired this year in skeletal programming, heterogeneous parallelism,
machine learning, and autotuning will all be required for achieving
this aim. The success of the project can be quantifiably measured in
terms of performance improvements to real world skeleton programs.


\label{bibliography}
\printbibliography

\end{document}
