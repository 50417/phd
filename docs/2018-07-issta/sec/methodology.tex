\section{Experimental Setup}

In this section we describe the experimental parameters used.

\subsection{OpenCL Systems}

We conducted testing of 10 OpenCL systems, summarized in Table~\ref{tab:platforms}.  We covered a broad range of hardware: 3 GPUs, 4 CPUs, a co-processor, and an emulator. 7 of the compilers tested are commercial products, 3 of them are open source. Our suite of systems includes both combinations of different drivers for the same device, and different devices using the same driver.

\subsection{Testbeds}
For each OpenCL system, we create two testbeds. In the first, the compiler is run with optimizations disabled. In the second, optimizations are enabled. Each testbed is then a triple, consisting of \emph{<device, driver, is\_optimized>} settings. This mechanism gives 20 testbeds to evaluate.

% \cc{TODO: 6 testbeds have not been tested by CLSmith}

\subsection{Test Cases}
For each generated program we create inputs as described in Section~\ref{sec:test-harness}. In addition, we need to choose the number of threads to use. We generate two test cases, one using one thread, the other using 2048 threads. A test case is then a triple, consisting of \emph{<program, inputs, threads>} settings.

\subsection{Bug Search Time Allowance}
We compare both our fuzzer and CLSmith. We allow both to run for 48 hours on each of the 20 testbeds.  CLSmith used its default configuration. The total runtime for a test case consists of the generation and execution time.
